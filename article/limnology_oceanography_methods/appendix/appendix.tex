\documentclass[12pt,a4paper]{scrartcl}
    \usepackage[utf8]{inputenc}
    \usepackage{amsmath}
    \usepackage{amsfonts}
    \usepackage{amssymb}
    \usepackage{graphicx}

    \usepackage[bottom = 1in, left = 0.5in, right = 0.5in, top = 1in]{geometry}

    \usepackage[english]{babel}
    \usepackage[autostyle]{csquotes}
    \usepackage{mathptmx}

    \usepackage[labelfont=bf]{caption}

    \usepackage[default, scale=0.95]{opensans}

    \usepackage[T1]{fontenc}

    \usepackage{fixltx2e}

    \addto\captionsenglish{\renewcommand{\figurename}{Supplementary Fig.}}
    \addto\captionsenglish{\renewcommand{\tablename}{Supplementary Table}}

    \title{Appendix}
    \date{}

	\PassOptionsToPackage{
	natbib=true,
	sorting=ynt,
	style=authoryear-comp,
	hyperref=true,
	backend=biber,
	maxbibnames=999,
	firstinits=true,
	uniquename=false,
	parentracker=true,
	url=false,
	doi=false,
	isbn=false,
	eprint=false,
	backref=false,
	sortcites,
}   {biblatex}
\usepackage{biblatex}

\DeclareLanguageMapping{english}{english-apa}
\addbibresource{/home/pmassicotte/Documents/library.bib}

\AtEveryBibitem{\clearfield{issn}}
\AtEveryCitekey{\clearfield{issn}}
\AtEveryBibitem{\clearfield{url}}
\AtEveryCitekey{\clearfield{url}}
\AtEveryBibitem{\clearfield{doi}}
\AtEveryCitekey{\clearfield{doi}}

\begin{document}
\maketitle

\section*{Sampling locations}

\begin{figure}[h]
	\centering
	\includegraphics[scale = 0.75]{../../../graphs/appendix_1.pdf}
	\caption{Locations of the ice stations sampled during the Transsiz expedition. Bathymetry data from the International Bathymetric Chart of the Arctic Ocean (IBCAO, v3.0).}
\end{figure}

\clearpage
\newpage

\section*{Incident light}

\begin{figure}[h]
	\centering
	\includegraphics[scale = 1]{../../../graphs/appendix_2.pdf}
	\caption{Hourly photosynthetic active radiation (PAR) measured at each station with a pyranometer installed onboard the ship. Numbers on top of the gray boxes identify the stations.}
\end{figure}

\clearpage
\newpage

\section*{Propagating light in the water column}

\begin{figure}[h]
	\centering
	\includegraphics[scale = 1]{../../../graphs/appendix_3.pdf}
	\caption{Propagated photosynthetic active radiation (PAR) in the water column at station 19 using ROV transmittance data. At this station, a total of 1561 transmittance values were measured by the ROV. Numbers on top of the gray boxes identify selected hours of the day. For visualization, data is plotted only between 0 and 10 meters.}
\end{figure}

\clearpage
\newpage

\section*{PvsE curves}

Two different models based on the original definition proposed by (Platt et al., 1980) were used depending on the situation.

\subsection*{Model with photoinhibition}

When apparent photo-inhibition was present, a model including two exponential was fitted (equation \ref{eq:two_exp}).

\begin{equation}
	p = ps \times (1 - e^{-\alpha \times \frac{PAR}{ps}}) \times e^{-\beta \times \frac{PAR}{ps}} + p0
	\label{eq:two_exp}
\end{equation}

\subsection*{Model without photoinhibition}

When no apparent photo-inhibition was present, a model including only one exponential was fitted (equation \ref{eq:one_exp}).

\begin{equation}
	p = ps \times (1 - e^{-\alpha \times \frac{PAR}{ps}}) + p0
	\label{eq:one_exp}
\end{equation}

The non-linear fitting was done using the Levenberg-Marquardt algorithm implemented in the minpack.lm R package \citep{Elzhov2013}.

\begin{figure}[h]
	\centering
	\includegraphics[scale = 01]{../../../graphs/appendix_4.pdf}
	\caption{Example of fitted PvsE curve using equation \ref{eq:one_exp}.}
\end{figure}

Using photosynthetic parameters derived from PvsE curves, primary production was calculated as:

\begin{equation}
	\text{Primary production} = ps \times (1 - e^{-\alpha \times \frac{PAR}{ps}})
	\label{eq:pp}
\end{equation}

\clearpage
\section*{Spatial autocorrelation}

\begin{figure}[h]
	\centering
	\includegraphics[scale = 1]{../../../graphs/appendix_5.pdf}
	\caption{Moran's I calculated from the SUIT transmittances ($0^-$).}
\end{figure}

\clearpage
\printbibliography

\end{document}
