%% March 2018
%DIF LATEXDIFF DIFFERENCE FILE
%DIF DEL ./initial_submission/pmassicotte_et_al_2019.tex   Mon Jan 28 15:16:55 2019
%DIF ADD ./revision_1/pmassicotte_et_al_2019.tex           Wed May 15 14:22:55 2019
%%%%%%%%%%%%%%%%%%%%%%%%%%%%%%%%%%%%%%%%%%%%%%%%%%%%%%%%%%%%%%%%%%%%%%%%%%%%
% AGUJournalTemplate.tex: this template file is for articles formatted with LaTeX
%
% This file includes commands and instructions
% given in the order necessary to produce a final output that will
% satisfy AGU requirements, including customized APA reference formatting.
%
% You may copy this file and give it your
% article name, and enter your text.
%
%
% Step 1: Set the \documentclass
%
% There are two options for article format:
%
% PLEASE USE THE DRAFT OPTION TO SUBMIT YOUR PAPERS.
% The draft option produces double spaced output.
%

%% To submit your paper:
\documentclass[draft]{agujournal2018}
\usepackage{apacite}
\usepackage{url} %this package should fix any errors with URLs in refs.
\usepackage{lineno}
\linenumbers
%DIF 27a27-29
 %DIF > 
\PassOptionsToPackage{hyphens}{url}\usepackage{hyperref} %DIF > 
 %DIF > 
%DIF -------
%%%%%%%
% As of 2018 we recommend use of the TrackChanges package to mark revisions.
% The trackchanges package adds five new LaTeX commands:
%
%  \note[editor]{The note}
%  \annote[editor]{Text to annotate}{The note}
%  \add[editor]{Text to add}
%  \remove[editor]{Text to remove}
%  \change[editor]{Text to remove}{Text to add}
%
% complete documentation is here: http://trackchanges.sourceforge.net/
%%%%%%%

\draftfalse

%% Enter journal name below.
%% Choose from this list of Journals:
%
% JGR: Atmospheres
% JGR: Biogeosciences
% JGR: Earth Surface
% JGR: Oceans
% JGR: Planets
% JGR: Solid Earth
% JGR: Space Physics
% Global Biogeochemical Cycles
% Geophysical Research Letters
% Paleoceanography and Paleoclimatology
% Radio Science
% Reviews of Geophysics
% Tectonics
% Space Weather
% Water Resources Research
% Geochemistry, Geophysics, Geosystems
% Journal of Advances in Modeling Earth Systems (JAMES)
% Earth's Future
% Earth and Space Science
% Geohealth
%
% ie, \journalname{Water Resources Research}

\journalname{JGR: Oceans}

\usepackage{textcomp}
\usepackage[utf8]{inputenc}
\usepackage{amsmath}

\usepackage{ragged2e}
\justifying

%% ------------------------------------------------------------------------ %%
%% Variables
%% ------------------------------------------------------------------------ %%

\newcommand{\ked}{\ensuremath{K_{E_d}}}
\newcommand{\klu}{\ensuremath{K_{L_u}}}
\newcommand{\edz}{\ensuremath{{E_d(z)}}}
\newcommand{\ed}{\ensuremath{{E_d}}}
%DIF 85a88
\newcommand{\lu}{\ensuremath{{L_u}}} %DIF > 
%DIF -------
\newcommand{\edzerominus}{\ensuremath{{E_d(0^-)}}}
\newcommand{\kdpar}{\ensuremath{K_{E_d}}(\textnormal{PAR})}
\newcommand{\kdparscalar}{\ensuremath{K_{\mathring{E}_d}}(\textnormal{PAR})}

\newcommand{\epar}{\ensuremath{E}(\textnormal{PAR})}
\newcommand{\eparz}{\ensuremath{E(\textnormal{PAR}, z)}}
\newcommand{\eparzint}{\ensuremath{E(\textnormal{PAR}, z\textsubscript{int})}}
\newcommand{\edlambdaz}{\ensuremath{{E_d(\lambda, z)}}}
\newcommand{\eparzero}{\ensuremath{E(\textnormal{PAR}, 0^+, t)}}

\newcommand{\eparscalar}{\ensuremath{\mathring{E}}(\textnormal{PAR})}
\newcommand{\eparzscalar}{\ensuremath{\mathring{E}(\textnormal{PAR}, z)}}
\newcommand{\eparzintscalar}{\ensuremath{\mathring{E}(\textnormal{PAR}, z\textsubscript{int})}}
\newcommand{\eparzeroscalar}{\ensuremath{\mathring{E}(\textnormal{PAR}, 0^+, t)}}
\newcommand{\eparzerosmoins}{\ensuremath{\mathring{E}(\textnormal{PAR}, 0^-)}}

\newcommand{\ppundericedevice}{\ensuremath{P_{\textnormal{\scriptsize underice}}^{\textnormal{\scriptsize device}}}}
\newcommand{\ppmixingdevice}{\ensuremath{P_{\textnormal{\scriptsize mixing}}^{\textnormal{\scriptsize device}}}}
\newcommand{\ppmixingsuit}{\ensuremath{P_{\textnormal{\scriptsize mixing}}^{\textnormal{\scriptsize SUIT}}}}
%DIF 104a108
\newcommand{\ppmixingrov}{\ensuremath{P_{\textnormal{\scriptsize mixing}}^{\textnormal{\scriptsize ROV}}}} %DIF > 
%DIF -------
\newcommand{\ppundericedevicezt}{\ensuremath{P_{\textnormal{\scriptsize underice}}^{\textnormal{\scriptsize device}}(z,t)}}
\newcommand{\ppsuitunderice}{\ensuremath{P_{\textnormal{\scriptsize underice}}^{\textnormal{\scriptsize SUIT}}}}
\newcommand{\pprovunderice}{\ensuremath{P_{\textnormal{\scriptsize underice}}^{\textnormal{\scriptsize ROV}}}}
\newcommand{\ppopenwater}{\ensuremath{P_{\textnormal{\scriptsize openwater}}}}
\newcommand{\ppmixing}{\ensuremath{P_{\textnormal{\scriptsize mixing}}}}
\newcommand{\ppunderice}{\ensuremath{P_{\textnormal{\scriptsize underice}}}}

%% ------------------------------------------------------------------------ %%
%%  Units
%% ------------------------------------------------------------------------ %%
\newcommand{\mminus}{m\textsuperscript{-1}}
\newcommand{\wmsquare}{W~m\textsuperscript{-2}}
\newcommand{\micromol}{\textmu mol~m\textsuperscript{-2}~s\textsuperscript{-1}}
\newcommand{\dailypp}{mgC~m\textsuperscript{-2}~d\textsuperscript{-1}}
%DIF PREAMBLE EXTENSION ADDED BY LATEXDIFF
%DIF UNDERLINE PREAMBLE %DIF PREAMBLE
\RequirePackage[normalem]{ulem} %DIF PREAMBLE
\RequirePackage{color}\definecolor{RED}{rgb}{1,0,0}\definecolor{BLUE}{rgb}{0,0,1} %DIF PREAMBLE
\providecommand{\DIFaddtex}[1]{{\protect\color{blue}\uwave{#1}}} %DIF PREAMBLE
\providecommand{\DIFdeltex}[1]{{\protect\color{red}\sout{#1}}}                      %DIF PREAMBLE
%DIF SAFE PREAMBLE %DIF PREAMBLE
\providecommand{\DIFaddbegin}{} %DIF PREAMBLE
\providecommand{\DIFaddend}{} %DIF PREAMBLE
\providecommand{\DIFdelbegin}{} %DIF PREAMBLE
\providecommand{\DIFdelend}{} %DIF PREAMBLE
\providecommand{\DIFmodbegin}{} %DIF PREAMBLE
\providecommand{\DIFmodend}{} %DIF PREAMBLE
%DIF FLOATSAFE PREAMBLE %DIF PREAMBLE
\providecommand{\DIFaddFL}[1]{\DIFadd{#1}} %DIF PREAMBLE
\providecommand{\DIFdelFL}[1]{\DIFdel{#1}} %DIF PREAMBLE
\providecommand{\DIFaddbeginFL}{} %DIF PREAMBLE
\providecommand{\DIFaddendFL}{} %DIF PREAMBLE
\providecommand{\DIFdelbeginFL}{} %DIF PREAMBLE
\providecommand{\DIFdelendFL}{} %DIF PREAMBLE
%DIF HYPERREF PREAMBLE %DIF PREAMBLE
\providecommand{\DIFadd}[1]{\texorpdfstring{\DIFaddtex{#1}}{#1}} %DIF PREAMBLE
\providecommand{\DIFdel}[1]{\texorpdfstring{\DIFdeltex{#1}}{}} %DIF PREAMBLE
%DIF LISTINGS PREAMBLE %DIF PREAMBLE
\RequirePackage{listings} %DIF PREAMBLE
\RequirePackage{color} %DIF PREAMBLE
\lstdefinelanguage{DIFcode}{ %DIF PREAMBLE
%DIF DIFCODE_UNDERLINE %DIF PREAMBLE
  moredelim=[il][\color{red}\sout]{\%DIF\ <\ }, %DIF PREAMBLE
  moredelim=[il][\color{blue}\uwave]{\%DIF\ >\ } %DIF PREAMBLE
} %DIF PREAMBLE
\lstdefinestyle{DIFverbatimstyle}{ %DIF PREAMBLE
	language=DIFcode, %DIF PREAMBLE
	basicstyle=\ttfamily, %DIF PREAMBLE
	columns=fullflexible, %DIF PREAMBLE
	keepspaces=true %DIF PREAMBLE
} %DIF PREAMBLE
\lstnewenvironment{DIFverbatim}{\lstset{style=DIFverbatimstyle}}{} %DIF PREAMBLE
\lstnewenvironment{DIFverbatim*}{\lstset{style=DIFverbatimstyle,showspaces=true}}{} %DIF PREAMBLE
%DIF END PREAMBLE EXTENSION ADDED BY LATEXDIFF

\begin{document}

%% ------------------------------------------------------------------------ %%
%  Title
%
% (A title should be specific, informative, and brief. Use
% abbreviations only if they are defined in the abstract. Titles that
% start with general keywords then specific terms are optimized in
% searches)
%
%% ------------------------------------------------------------------------ %%

% Example: \title{This is a test title}

\title{Sensitivity of phytoplankton primary production estimates to available irradiance under heterogeneous sea-ice conditions}

%% ------------------------------------------------------------------------ %%
%
%  AUTHORS AND AFFILIATIONS
%
%% ------------------------------------------------------------------------ %%

% Authors are individuals who have significantly contributed to the
% research and preparation of the article. Group authors are allowed, if
% each author in the group is separately identified in an appendix.)

% List authors by first name or initial followed by last name and
% separated by commas. Use \affil{} to number affiliations, and
% \thanks{} for author notes.
% Additional author notes should be indicated with \thanks{} (for
% example, for current addresses).

% Example: \authors{A. B. Author\affil{1}\thanks{Current address, Antartica}, B. C. Author\affil{2,3}, and D. E.
% Author\affil{3,4}\thanks{Also funded by Monsanto.}}

\authors{Philippe Massicotte\affil{1,5}, Ilka Peeken\affil{2}, Christian Katlein\affil{1,2}, Hauke Flores\affil{2}, Yannick Huot\affil{3}, Giulia Castellani\affil{2}, Stefanie Arndt\affil{2}, Benjamin A. Lange\affil{2,4}, Jean-Éric Tremblay\affil{1,5} and Marcel Babin\affil{1,5}}

\affiliation{1}{Takuvik Joint International Laboratory (UMI 3376) -- Université Laval (Canada) \& Centre National de la Recherche Scientifique (France)}
\affiliation{2}{Alfred-Wegener-Institut Helmholtz-Zentrum für Polar- und Meeresforschung, Bremerhaven, Germany}
\affiliation{3}{Université de Sherbrooke, Sherbrooke, Québec, Canada, J1K 2R1}
\affiliation{4}{Fisheries and Oceans Canada, Freshwater Institute, Winnipeg, MB, Canada}
\affiliation{5}{Québec-Océan et département de biologie, Université Laval, Québec, Canada, G1V 0A6}

%(repeat as many times as is necessary)

%% Corresponding Author:
% Corresponding author mailing address and e-mail address:

% (include name and email addresses of the corresponding author.  More
% than one corresponding author is allowed in this LaTeX file and for
% publication; but only one corresponding author is allowed in our
% editorial system.)

% Example: \correspondingauthor{First and Last Name}{email@address.edu}

\correspondingauthor{Philippe Massicotte}{philippe.massicotte@takuvik.ulaval.ca}

%% Keypoints, final entry on title page.

%  List up to three key points (at least one is required)
%  Key Points summarize the main points and conclusions of the article
%  Each must be 100 characters or less with no special characters or punctuation

% Example:
% \begin{keypoints}
% \item	List up to three key points (at least one is required)
% \item	Key Points summarize the main points and conclusions of the article
% \item	Each must be 100 characters or less with no special characters or punctuation
% \end{keypoints}

\begin{keypoints}
	\item Phytoplankton primary production under heterogeneous sea ice is highly spatially variable.
	\item Transmittance sampled with profiling platforms improves the accuracy of primary production estimates.
	\item Upscaling estimates at larger spatial scales using satellite sea-ice concentration further reduced the error.
\end{keypoints}

%% ------------------------------------------------------------------------ %%
%
%  ABSTRACT
%
% A good abstract will begin with a short description of the problem
% being addressed, briefly describe the new data or analyses, then
% briefly states the main conclusion(s) and how they are supported and
% uncertainties.
%% ------------------------------------------------------------------------ %%

%% \begin{abstract} starts the second page

\begin{abstract}

	The Arctic icescape is \DIFdelbegin \DIFdel{becoming an increasingly complex mosaic composed }\DIFdelend \DIFaddbegin \DIFadd{composed by a mosaic }\DIFaddend of ridges, hummocks, melt ponds, leads and snow. Under such heterogeneous surfaces, drifting phytoplankton communities are experiencing a wide range of irradiance conditions and intensities that cannot be sampled representatively using \DIFdelbegin \DIFdel{single-point }\DIFdelend \DIFaddbegin \DIFadd{single-location }\DIFaddend measurements. Combining experimentally derived photosynthetic parameters with transmittance measurements acquired at spatial scales ranging from hundreds of meters (using a Remotely Operated Vehicle, ROV) to thousands of meters (using a Surface and Under-Ice Trawl, SUIT), we assessed the sensitivity of water-column primary production estimates to multi-scale under-ice light measurements. Daily primary production calculated from transmittance from both the ROV and the SUIT ranged between 0.004 and 939 \dailypp{}. Upscaling these estimates at larger spatial scales using satellite-derived sea-ice concentration reduced the variability by 22\% (0.004-731 \dailypp{}). The relative error in primary production estimates was two times lower when combining remote sensing and in situ data compared to ROV-based estimates alone. These results suggest that spatially extensive in situ measurements must be combined with large-footprint sea-ice coverage sampling (e.g., remote sensing, aerial imagery) to accurately estimate primary production in ice-covered waters. Also, the results indicated a decreasing error of primary production estimates with increasing sample size and the spatial scale \DIFdelbegin \DIFdel{of }\DIFdelend \DIFaddbegin \DIFadd{at which }\DIFaddend in situ measurements \DIFaddbegin \DIFadd{are performed}\DIFaddend . Conversely, existing estimates of spatially integrated phytoplankton primary production in ice-covered waters \DIFdelbegin \DIFdel{using single-point }\DIFdelend \DIFaddbegin \DIFadd{derived  from single-location light }\DIFaddend measurements may be associated with large statistical errors. Considering these implications is important for modelling scenarios and interpretation of existing measurements in a changing Arctic ecosystem. 

\end{abstract}

%% ------------------------------------------------------------------------ %%
%
%  TEXT
%
%% ------------------------------------------------------------------------ %%

%%% Suggested section heads:
% \section{Introduction}
%
% The main text should start with an introduction. Except for short
% manuscripts (such as comments and replies), the text should be divided
% into sections, each with its own heading.

% Headings should be sentence fragments and do not begin with a
% lowercase letter or number. Examples of good headings are:

% \section{Materials and Methods}
% Here is text on Materials and Methods.
%
% \subsection{A descriptive heading about methods}
% More about Methods.
%
% \section{Data} (Or section title might be a descriptive heading about data)
%
% \section{Results} (Or section title might be a descriptive heading about the
% results)
%
% \section{Conclusions}

\section{Introduction}
The Arctic sea icescape is characterized by a mosaic composed of sea ice, snow, leads, melt ponds and open water. During the last decades, this arctic icescape has been undergoing major changes, including a reduction of sea ice cover and thickness \citep{Meier2014}, and increased drift speed (Kwok2013). Increasing storm events is also making this icescape more prone to deformation (Itkin2017) and increase the frequency of lead formations. Because of this surface heterogeneity, light transmittance can be highly variable in space, even over short distances (Nicolaus2013b, Katlein2015, Hancke2018). For example, Perovich1998 showed that ice and snow transmittance at 440 nm could vary by a factor of two over horizontal distances of 25 m. The relative contribution of various sea-ice features to under-ice light variability depends on the spatial scale under consideration and has significant implications for their application in physical and ecological studies and also determines the context in which results can be interpreted. For instance, at small scales (< 100 m), local features such as melt ponds and leads have a strong influence on light penetration and fluctuations \citep{Frey2011, Katlein2016, Massicotte2018}. At larger scales (> 100 m), it was argued that the variability of transmittance is mainly controlled by sea ice thickness \citep{Katlein2015}.
\section{Materials and Methods}

\begin{linenomath*}
\begin{equation}
    T(z_\textnormal{int}) = \frac{T(z)}{e^{-K_d(PAR)-z}}
\end{equation}

\noindent where $T(z_\textnormal{int})$ is the transmittance of the ice and snow at the ice-water interface, 

\end{linenomath*}
\section{Results}

\subsection{Characterization of the sea-ice and snow cover}

GEM-2 and Magna Probe surveys along and across the ROV transects showed distinct differences in sea ice and snow thickness between the sampled stations. An overview of the total thickness (i.e., combined snow and ice thickness) is presented in Figure 2A. Overall, the mean ice thickness was 1.01 $\pm$ 0.52 m (mean $\pm$ s.d.), the mean snow thickness was 0.32 $\pm$ 0.16 m and the mean total thickness was 1.33 $\pm$ 0.49 m (Figure 2B). Stations 19 and 47 were characterized by an average total thickness over the ROV transect of approximately 1 m, whereas the average total thickness at station 39 was approximately 2 m. For other stations, average total thickness varied around 1.4 m.

\subsection{ROV and SUIT transmittance measurements}

A total of 9211 and 817 transmittance measurements distributed over the seven stations were collected from the ROV and SUIT devices, respectively (Figure 3). Transmittance values ranged between 0.001\% and 68\% for the ROV and between 0.002\% and 92\% for the SUIT (Figure 3). The transmittances measured by the SUIT were generally higher (mean = 35\%) by approximately one order magnitude than those measured with the ROV (mean = 2\%). The SUIT measurements were also covering greater ranges of transmittances compared to the ROV. Histograms showed that transmittance generally followed a bimodal distribution (most of the time occurring within the SUIT data) with often one overlapping mode between the ROV and SUIT values (Figure 3). 

\subsection{Photosynthetically active radiation (PAR)}

Incident hourly \eparscalar{}, \eparzeroscalar{}, measured by the pyranometer ranged between 190 and 1400 \micromol{} (Figure 4). Stations 32 and 39 experienced the highest incident \eparzeroscalar{} whereas stations 27 and 43 received the lowest amount of light. Over 24h periods, \eparzintscalar{} calculated using SUIT and ROV transmittances ranged between 0.005-1358 and 0.005-1012 \micromol{} respectively. Due to relatively high attenuation coefficients (Table 1), \eparscalar{} decreased rapidly with depth and generally reached the asymptotic regime at maximum 30 m depth. The PAR diffuse vertical attenuation coefficients, \kdparscalar{}, estimated from the ROV vertical profiles varied between 0.07 and 0.59 m\textsuperscript{-1} (Table 1).

\subsection{Estimated primary production}

Daily areal primary production derived from photosynthetic parameters and transmittance values ranged between 0.004 and 939 \dailypp{} for \ppunderice{} and between 0.004 and 731 \dailypp{} for \ppmixing{} (Figure 5). In \DIFdelbegin \DIFdel{ROV-bases }\DIFdelend \DIFaddbegin \DIFadd{ROV-based }\DIFaddend estimates, daily areal primary productions calculated using the two different approaches (\ppunderice{} and \ppmixing{}) generally showed consistency especially when SIC was high. At stations 19 and 27, greater differences between \ppunderice{} and \ppmixing{} were observed in ROV-based estimates due to lower sea ice concentrations (Table 1) which allowed for a greater weight of \ppopenwater{} on the calculations. In SUIT-based estimates, mean daily \ppunderice{} values were higher than \ppmixing{} values at stations 19, 39 and 43, similar \DIFdelbegin \DIFdel{values }\DIFdelend at stations 27, 46 and 47, and lower \DIFdelbegin \DIFdel{values }\DIFdelend at station 31 (Figure 5). The \DIFaddbegin \DIFadd{10\% transmittance threshold used to filter out SUIT-based data explains why mean values of daily }\ppunderice{} \DIFadd{can be lower than those of based on ROV measurements. The }\DIFaddend differences between the two approaches in SUIT data were related to the varying proportions of thin ice and open water during SUIT hauls, which were reflected in the \ppunderice{} estimates. Overall, both ROV- and SUIT based estimates agreed well with each other when the mixing approach (\ppmixing{}) was applied.

\subsection{Error on primary production estimates}

Figure 6 shows the distributions of the relative errors around the calculated average of areal primary production (see black dots in Figure 5). Overall, the absolute relative errors ($\delta_P$) were distributed over a range covering four orders of magnitude, between 0.1\% and 1000\% which is corresponding to absolute primary production error varying between 0.0001 and 640 \dailypp{}. The lowest absolute errors (average $\approx$50\%) were associated with primary production estimates made using the mixing model approach (\ppmixing{}). Larger absolute errors were made with \ppunderice{} derived from only using ROV (mean = 88\%) and the SUIT (mean = 71\%) transmittances.

\subsection{Impacts of the number of in situ \DIFdelbegin \DIFdel{spot }\DIFdelend \DIFaddbegin \DIFadd{light }\DIFaddend measurements on primary production estimates}

Figure 7 shows the average relative error that one would make when averaging \DIFdelbegin \DIFdel{samples }\DIFdelend \DIFaddbegin \DIFadd{light measurements performed }\DIFaddend at a number of random locations varying between 1 and 250. \DIFdelbegin \DIFdel{For all scenarios, the mean relative error decreased exponentially with increasing number of chosen observations. }\DIFdelend The variability around the means also decreased with increasing number of observations (shaded areas in Figure 7). The greatest relative mean error ($\approx$60-100\%) occurred when only one primary production estimate was randomly selected from the distributions. The number of randomly selected observations to reach mean relative errors of 10\%, 15\%, 20\% and 25\% are presented in Table 3. Overall, about 25\% the number of observations were needed to reach those targets when sampling from the distribution for \ppmixing{} compared to the distribution of \ppunderice{}. Additionally, the number of observations required when using the SUIT transmittance to derive primary production estimation was also about 25\% of the number of corresponding ROV-based measurements to reach the same error threshold.
\section{Discussion}

\subsection{Multi-scale spatial variability of light transmittance}

In the context of obtaining meaningful measurements of transmittance to accurately estimate \eparzerosmoins{}, the challenge is to define the spatial extent at which light should be sampled. Based on a spatial autocorrelation analysis conducted in the central Arctic ocean, it was determined that transmittance values were uncorrelated (i.e., randomly spatially distributed) to each other after a horizontal lag distance of 65 m \citep{Lange2017b}. This range is likely to be much smaller than the distance covered by drifting phytoplankton over a 24h period. Indeed, water currents around Svalbard have been found to vary between 0.14 and 0.21 m s\textsuperscript{-1} \citep{Meyer2017} These speeds are on the same order of magnitude as the sea ice drift speeds of 0.10. m s \textsuperscript{-1} observed during the expedition. Assuming passive transport, this corresponds to a displacement varying between 8 and 18 km over a 24h period which is much greater than the 65 m distance at which transmittance was found to be randomly spatially distributed. Under such a large area, drifting phytoplankton is experiencing a wide range of irradiance conditions that can be hardly characterized by single-spot measurements or even with ROV and SUIT devices sampling over larger distances. In such context, measured transmittances should be upscaled at the spatial scale that is meaningful for the studied process. An easily applicable approach to upscale in-situ transmittance measurements consists of using sea ice concentration (SIC) derived from satellite imagery. A simple mixing model (equation 6), combining both in-situ transmittance measurements and SIC, can be used to upscale observations acquired locally to larger scales. Our results showed that using this approach reduced the relative error by approximately a factor of two when spatially integrating devices such as ROVs or SUIT are used to measure transmittance (Figure 5). Furthermore, this error was lower when using in-situ measurements acquired on a larger spatial scale using the SUIT. This strengthens the idea that one needs to characterize the light field over an area as large as reasonably possible so the full irradiance variability is captured.

Our study confirms earlier suggestions that estimating primary production from photosynthetic parameters and transmittance measured at a single location does not provide a representative description of the spatial variability of the primary production occurring under a heterogeneous sea surface (Figure 6, Figure 7). Depending on the scale at which transmittance was measured, it was found that deriving primary production from photosynthetic parameters using under-ice profile  measurements alone would produce on average relative errors varying between 47\% and 88\% (Figure 6). In contrast, much lower errors (25\%) were made when primary production estimates were upscaled using satellite-derived SIC (\ppmixing{}). For stations with lower SIC (stations 19, 27, 31 and 39), primary production estimates were more constrained around the average (Figure 4) because \ppopenwater{} had a greater weight in the calculation of \ppmixing{} (see equation 5). For stations 43, 46 and 47 where SIC was 100\%, the spread around the mean was higher because only \ppunderice{} was contributing to the calculation of \ppmixing{}. These results suggest that using a distribution of measured transmittances allows calculating a more representative transmittance average for a given area, but also provides additional knowledge on its spatial variability.

Although our results indicate that it is necessary to properly characterize the light field under the heterogeneous sea surface, the physiological state of the phytoplankton community also plays a major role on the sensitivity of the estimates to incoming irradiance. An important parameter of the physiological state of the phytoplankton community is the light-saturated photosynthesis regime, $E_k$ an index of photoadaptation. When the phytoplankton community is adapted to low light intensity \citep [e.g.,][]{Lacour2017}, it is likely that variations in the surface light field have reduced impacts on the estimates because phytoplankton primary production is already near saturation. The degree of photoadaptation of the phytoplankton communities and their ability to adjust rapidly to a variable light field still remains to be evaluated.

\subsection{Influence of the number of sampling locations on primary production estimation}

It was pointed out by \citet{Nicolaus2013} that it is difficult to characterize light conditions under sea ice over large areas and to quantify spatial variability on different scales due to important requirements in logistical and instrumental efforts. As with any missions in remote environments such as the Arctic, careful planning is needed to find the right balance between the sampling effort and the right amount of acquired information to study a particular phenomenon. Our results suggested that errors made by estimating primary production using photosynthetic parameters decreased exponentially with increasing number of transmittance measurements (Figure 7). Depending on the extent of the spatial scale at which transmittance is measured (order of meters for the ROV, order of kilometres for the SUIT) and the targeted error thresholds (10\%, 15\%, 20\% or 25\%), a number of measurements varying between four and 359 were sufficient to reasonably capture the spatial variability of sea ice transmittance to derive average primary production estimates over a given area. This shows, that local primary production estimated from just a single or even a handful of light observations has limited value.

\subsection{Implications for Arctic primary production estimates}

It is known that the annual primary production in the ice-covered Arctic is among the lowest of all oceans worldwide because both light and limited nutrient availability are the main limiting factors for phytoplankton growth under the ice. In a changing Arctic icescape, efforts have been devoted to better understand how phytoplankton primary productivity is responding to increasing light availability. Many studies have been conducted in the vicinity of an ice edge to characterize primary production occurring under the ice sheet \citep{Arrigo2012, Arrigo2014, Mundy2009}. However, in such studies, due to logistical constraints, the underwater light field wasoften characterized by a limited number of light measurements. Other approaches, based on 24h ship-board incubations performed under incident light, have provided local estimates that were simply scaled to an assessment of percent ice-cover in the vicinity of the ship \citep{Smith1995, Gosselin1997, Mei2003}. Therefore, depending on whether light is measured under bare ice or in open water, the estimated primary production is either under- or overestimated. Different approaches based on remote sensing techniques and modelling have been used to reduce the high uncertainties associated with estimates derived from local in-situ measurements. However, in an ecosystem model intercomparison study, \citet{Jin2015} showed that under-ice primary production was very sensitive to the light availability computed by atmospheric and sea ice models, reinforcing the need to develop new integrative strategies to adequately characterize the light field at large scale under heterogeneous sea surfaces. Our results showed that upscaling primary production estimates derived from fine-scale local measurements using SIC derived from satellite imagery allowed reducing the error at larger spatial scales. Furthermore, it was found that even when SIC was high (\textgreater 95\%), the use of a mixing-model approach helped to obtain better estimates (Figure 5).
\section{Conclusions}

Advances in underwater technologies made it easier to characterize surface transmittance over large areas. Our results showed that combining photosynthetic parameters measured in laboratory experiments with spatially representative transmittance values sampled with under-ice profiling platforms can significantly improve the accuracy of primary production estimates under heterogeneous sea surfaces. A good way forward to sample the under-ice light field on a large enough scale without the inherent biases of the ROV and SUIT deployment techniques would be the use of long-range autonomous underwater vehicles. Furthermore, upscaling in-situ measurements at larger scales using remote sensing data becomes necessary when the spatial scale of the studied process (e.g., a phytoplankton bloom) is greater than that which is realistically possible to measure in the field. This emphasizes the need for spatially integrated observation approaches to characterize the light field in ice-covered regions in order to provide more representative primary production estimates.

%  Numbered lines in equations:
%  To add line numbers to lines in equations,
%  \begin{linenomath*}
%  \begin{equation}
%  \end{equation}
%  \end{linenomath*}



%% Enter Figures and Tables near as possible to where they are first mentioned:
%
% DO NOT USE \psfrag or \subfigure commands.
%
% Figure captions go below the figure.
% Table titles go above tables;  other caption information
%  should be placed in last line of the table, using
% \multicolumn2l{$^a$ This is a table note.}
%
%----------------
% EXAMPLE FIGURE
%
% \begin{figure}[h]
% \centering
% when using pdflatex, use pdf file:
% \includegraphics[natwidth=800px,natheight=600px]{figsamp.pdf}
%
% when using dvips, use .eps file:
% \includegraphics[natwidth=800px,natheight=600px]{figsamp.eps}
%
% \caption{Short caption}
% \label{figone}
%  \end{figure}
%
% We recommend that you provide the native width and height (natwidth, natheight) of your figures.
% Specifying native dimensions ensures that your figures are properly scaled
%
%
% ---------------
% EXAMPLE TABLE
%
% \begin{table}
% \caption{Time of the Transition Between Phase 1 and Phase 2$^{a}$}
% \centering
% \begin{tabular}{l c}
% \hline
%  Run  & Time (min)  \\
% \hline
%   $l1$  & 260   \\
%   $l2$  & 300   \\
%   $l3$  & 340   \\
%   $h1$  & 270   \\
%   $h2$  & 250   \\
%   $h3$  & 380   \\
%   $r1$  & 370   \\
%   $r2$  & 390   \\
% \hline
% \multicolumn{2}{l}{$^{a}$Footnote text here.}
% \end{tabular}
% \end{table}

%% SIDEWAYS FIGURE and TABLE
% AGU prefers the use of {sidewaystable} over {landscapetable} as it causes fewer problems.
%
% \begin{sidewaysfigure}
% \includegraphics[width=20pc]{figsamp}
% \caption{caption here}
% \label{newfig}
% \end{sidewaysfigure}
%
%  \begin{sidewaystable}
%  \caption{Caption here}
% \label{tab:signif_gap_clos}
%  \begin{tabular}{ccc}
% one&two&three\\
% four&five&six
%  \end{tabular}
%  \end{sidewaystable}

%% If using numbered lines, please surround equations with \begin{linenomath*}...\end{linenomath*}
%\begin{linenomath*}
%\begin{equation}
%y|{f} \sim g(m, \sigma),
%\end{equation}
%\end{linenomath*}

%%% End of body of article

%%%%%%%%%%%%%%%%%%%%%%%%%%%%%%%%
%% Optional Appendix goes here
%
% The \appendix command resets counters and redefines section heads
%
% After typing \appendix
%
%\section{Here Is Appendix Title}
% will show
% A: Here Is Appendix Title
%
%\appendix
%\section{Here is a sample appendix}

%%%%%%%%%%%%%%%%%%%%%%%%%%%%%%%%%%%%%%%%%%%%%%%%%%%%%%%%%%%%%%%%
%
% Optional Glossary, Notation or Acronym section goes here:
%
%%%%%%%%%%%%%%
% Glossary is only allowed in Reviews of Geophysics
%  \begin{glossary}
%  \term{Term}
%   Term Definition here
%  \term{Term}
%   Term Definition here
%  \term{Term}
%   Term Definition here
%  \end{glossary}

%
%%%%%%%%%%%%%%
% Acronyms
%   \begin{acronyms}
%   \acro{Acronym}
%   Definition here
%   \acro{EMOS}
%   Ensemble model output statistics
%   \acro{ECMWF}
%   Centre for Medium-Range Weather Forecasts
%   \end{acronyms}

%
%%%%%%%%%%%%%%
% Notation
%   \begin{notation}
%   \notation{$a+b$} Notation Definition here
%   \notation{$e=mc^2$}
%   Equation in German-born physicist Albert Einstein's theory of special
%  relativity that showed that the increased relativistic mass ($m$) of a
%  body comes from the energy of motion of the body—that is, its kinetic
%  energy ($E$)—divided by the speed of light squared ($c^2$).
%   \end{notation}




%%%%%%%%%%%%%%%%%%%%%%%%%%%%%%%%%%%%%%%%%%%%%%%%%%%%%%%%%%%%%%%%
%
%  ACKNOWLEDGMENTS
%
% The acknowledgments must list:
%
% >>>>	A statement that indicates to the reader where the data
% 	supporting the conclusions can be obtained (for example, in the
% 	references, tables, supporting information, and other databases).
%
% 	All funding sources related to this work from all authors
%
% 	Any real or perceived financial conflicts of interests for any
%	author
%
% 	Other affiliations for any author that may be perceived as
% 	having a conflict of interest with respect to the results of this
% 	paper.
%
%
% It is also the appropriate place to thank colleagues and other contributors.
% AGU does not normally allow dedications.

\acknowledgments

We thank F. Bruyant, M. Beaulieu for carrying out the P vs. E curve measurements and providing us with the data. We thank Sascha Willmes for onboard processing of the ice and snow thickness data. We thank captain Thomas Wunderlich and the crew of icebreaker Polarstern for their support during the TRANSSIZ campaign (AWI\_PS92\_00). This study was conducted under the Helmholtz Association Research Programme Polar regions And Coasts in the changing Earth System II (PACES II), Topic 1, WP 4 and is part of the Helmholtz Association Young Investigators Groups Iceflux: Ice-ecosystem carbon flux in polar oceans (VH-NG-800). \DIFaddbegin \DIFadd{BAL was partly funded during this study by a Visiting Fellowship from the Natural Sciences and Engineering Research Council of Canada (NSERC). The project was conducted under the scientific coordination of the Canada Excellence Research Chair on Remote sensing of Canada's new Arctic frontier and the CNRS \& Université Laval Takuvik Joint International laboratory (UMI3376). We also acknowledge the Sentinel North Strategy for their financial support. }\DIFaddend SUIT was developed by Wageningen Marine Research (WMR; formerly IMARES) with support from the Netherlands Ministry of EZ (project WOT-04-009-036) and the Netherlands Polar Program (project ALW 866.13.009). We thank Jan Andries van Franeker (WMR) for kindly providing the Surface and Under Ice Trawl (SUIT) and Michiel van Dorssen for technical support with work at sea. \DIFdelbegin \DIFdel{The project was conducted under the scientific coordination of the Canada Excellence Research Chair on Remote sensing of Canada's new Arctic frontier and the CNRS \& Université Laval Takuvik Joint International laboratory (UMI3376). }\DIFdelend Data for the light measurement used in this study can be found on Pangaea website\DIFdelbegin \DIFdel{: 
}%DIFDELCMD < 

%DIFDELCMD < \begin{itemize}
%DIFDELCMD < 	\item %%%
\DIFdelend \DIFaddbegin \DIFadd{. }\DIFaddend ROV data (\DIFdelbegin \DIFdel{https://doi.pangaea.de/10.1594/PANGAEA.861048)}%DIFDELCMD < \item %%%
\DIFdel{Incident radiation (https://doi.pangaea.de/10.1594/PANGAEA.849663)}%DIFDELCMD < \item %%%
\DIFdel{Station list (}\DIFdelend \DIFaddbegin \url{https://doi.pangaea.de/10.1594/PANGAEA.861048}\DIFadd{), incident radiation (}\url{https://doi.pangaea.de/10.1594/PANGAEA.849663}\DIFadd{), station list (}\url{https://doi.pangaea.de/10.1594/PANGAEA.848841}\DIFadd{), SUIT data (submitted to PANGAEA (}\DIFaddend https://doi.\DIFdelbegin \DIFdel{pangaea.de}\DIFdelend \DIFaddbegin \DIFadd{org}\DIFaddend /10.1594/\DIFdelbegin \DIFdel{PANGAEA.848841) , 
	}%DIFDELCMD < \item %%%
\DIFdel{SUIT data (submitted to Pangaea)}%DIFDELCMD < \item %%%
\DIFdel{Photosynthetic parameters (submitted to Pangaea) }%DIFDELCMD < \item %%%
\DIFdel{Sea-ice}\DIFdelend \DIFaddbegin \DIFadd{pangaea) and are in the curation process. It will be available in open access shortly.), photosynthetic parameters (}\url{https://doi.org/10.1594/PANGAEA.899842}\DIFadd{) and sea-ice}\DIFaddend /snow thickness (\DIFdelbegin \DIFdel{submitted to Pangaea)}%DIFDELCMD < \end{itemize}
%DIFDELCMD < %%%
\DIFdelend \DIFaddbegin \url{https://doi.pangaea.de/10.1594/PANGAEA.897958}\DIFadd{).
}\DIFaddend 

%% ------------------------------------------------------------------------ %%
%% References and Citations

%%%%%%%%%%%%%%%%%%%%%%%%%%%%%%%%%%%%%%%%%%%%%%%
% BibTeX is preferred:
%
\bibliography{/home/pmassicotte/Documents/library}
%
% don't specify bibliographystyle
%%%%%%%%%%%%%%%%%%%%%%%%%%%%%%%%%%%%%%%%%%%%%%%



% Please use ONLY \citet and \citep for reference citations.
% DO NOT use other cite commands (e.g., \cite, \citeyear, \nocite, \citealp, etc.).
%% Example \citet and \citep:
%  ...as shown by \citet{Boug10}, \citet{Buiz07}, \citet{Fra10},
%  \citet{Ghel00}, and \citet{Leit74}.

%  ...as shown by \citep{Boug10}, \citep{Buiz07}, \citep{Fra10},
%  \citep{Ghel00, Leit74}.

%  ...has been shown \citep [e.g.,][]{Boug10,Buiz07,Fra10}.


\end{document}



More Information and Advice:

%% ------------------------------------------------------------------------ %%
%
%  SECTION HEADS
%
%% ------------------------------------------------------------------------ %%

% Capitalize the first letter of each word (except for
% prepositions, conjunctions, and articles that are
% three or fewer letters).

% AGU follows standard outline style; therefore, there cannot be a section 1 without
% a section 2, or a section 2.3.1 without a section 2.3.2.
% Please make sure your section numbers are balanced.
% ---------------
% Level 1 head
%
% Use the \section{} command to identify level 1 heads;
% type the appropriate head wording between the curly
% brackets, as shown below.
%
%An example:
%\section{Level 1 Head: Introduction}
%
% ---------------
% Level 2 head
%
% Use the \subsection{} command to identify level 2 heads.
%An example:
%\subsection{Level 2 Head}
%
% ---------------
% Level 3 head
%
% Use the \subsubsection{} command to identify level 3 heads
%An example:
%\subsubsection{Level 3 Head}
%
%---------------
% Level 4 head
%
% Use the \subsubsubsection{} command to identify level 3 heads
% An example:
%\subsubsubsection{Level 4 Head} An example.
%
%% ------------------------------------------------------------------------ %%
%
%  IN-TEXT LISTS
%
%% ------------------------------------------------------------------------ %%
%
% Do not use bulleted lists; enumerated lists are okay.
% \begin{enumerate}
% \item
% \item
% \item
% \end{enumerate}
%
%% ------------------------------------------------------------------------ %%
%
%  EQUATIONS
%
%% ------------------------------------------------------------------------ %%

% Single-line equations are centered.
% Equation arrays will appear left-aligned.

Math coded inside display math mode \[ ...\]
will not be numbered, e.g.,:
\[ x^2=y^2 + z^2\]

Math coded inside \begin{equation} and \end{equation} will
be automatically numbered, e.g.,:
\begin{equation}
	x^2=y^2 + z^2
\end{equation}


% To create multiline equations, use the
% \begin{eqnarray} and \end{eqnarray} environment
% as demonstrated below.
\begin{eqnarray}
	x_{1} & = & (x - x_{0}) \cos \Theta \nonumber \\
	&& + (y - y_{0}) \sin \Theta  \nonumber \\
	y_{1} & = & -(x - x_{0}) \sin \Theta \nonumber \\
	&& + (y - y_{0}) \cos \Theta.
\end{eqnarray}

%If you don't want an equation number, use the star form:
%\begin{eqnarray*}...\end{eqnarray*}

% Break each line at a sign of operation
% (+, -, etc.) if possible, with the sign of operation
% on the new line.

% Indent second and subsequent lines to align with
% the first character following the equal sign on the
% first line.

% Use an \hspace{} command to insert horizontal space
% into your equation if necessary. Place an appropriate
% unit of measure between the curly braces, e.g.
% \hspace{1in}; you may have to experiment to achieve
% the correct amount of space.


%% ------------------------------------------------------------------------ %%
%
%  EQUATION NUMBERING: COUNTER
%
%% ------------------------------------------------------------------------ %%

% You may change equation numbering by resetting
% the equation counter or by explicitly numbering
% an equation.

% To explicitly number an equation, type \eqnum{}
% (with the desired number between the brackets)
% after the \begin{equation} or \begin{eqnarray}
% command.  The \eqnum{} command will affect only
% the equation it appears with; LaTeX will number
% any equations appearing later in the manuscript
% according to the equation counter.
%

% If you have a multiline equation that needs only
% one equation number, use a \nonumber command in
% front of the double backslashes (\\) as shown in
% the multiline equation above.

% If you are using line numbers, remember to surround
% equations with \begin{linenomath*}...\end{linenomath*}

%  To add line numbers to lines in equations:
%  \begin{linenomath*}
%  \begin{equation}
%  \end{equation}
%  \end{linenomath*}



