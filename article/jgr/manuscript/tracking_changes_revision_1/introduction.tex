\section{Introduction}

The Arctic sea icescape is characterized by a mosaic composed of sea ice, snow, leads, melt ponds and open water. During the last decades, this arctic icescape has been undergoing major changes, including a reduction of sea ice cover and thickness \citep{Meier2014}, and increased drift speed \citep{Kwok2013}. A greater frequency of storm events is also making this icescape more prone to deformation \citep{Itkin2017} and promotes lead formation. Because of this surface heterogeneity, light transmittance can be highly variable in space, even over short distances \citep{Nicolaus2013b, Katlein2015, Hancke2018}. For example, \citet{Perovich1998} showed that ice and snow transmittance at 440 nm could vary by a factor of two over horizontal distances of 25 m. The relative contribution of various sea-ice features to under-ice light variability depends on the spatial scale under consideration and has significant implications for their application in physical and ecological studies and also determines the context in which results can be interpreted. For instance, at small scales (\textless 100 m), local features such as melt ponds and leads have a strong influence on light penetration and fluctuations \citep{Frey2011, Katlein2016, Massicotte2018}. At larger scales (\textgreater 100 m), it was argued that the variability of transmittance is mainly controlled by sea ice thickness \citep{Katlein2015}.

Calculation of primary production based on incubations or photosynthetic parameters derived from photosynthesis vs. irradiance curves (P vs. E curves) requires adequately measured or estimated values of irradiance. Because phytoplankton is exposed to a highly variable light regime while drifting under a spatially heterogeneous, and sometimes dynamic sea-ice surface, local irradiance measurements are not representative of the average irradiance experienced by phytoplankton over a large area \citep{Katlein2016, Lange2017}. One major challenge in obtaining adequate irradiance estimates under spatially heterogeneous sea ice is that observations are often limited to time-consuming spot measurements made through boreholes. To overcome this drawback, different underwater technologies have been developed to study the spatial variability of light transmission under spatially heterogeneous sea surfaces. 

For the last decade, radiometers have been attached to remotely operated vehicles (ROV). Small sized ROVs can be deployed through small holes (\textless 2 m) to cover areas in the order of a few hundreds of meters \citep{Katlein2015, Katlein2017, Ambrose2005, Lund-Hansen2018, Nicolaus2010}. Navigating directly under sea ice, ROVs allow covering various types of sea ice, such as newly formed, ponded and snow-covered sea ice, as well as pressure ridges \citep{Katlein2017}. More recently, radiometers have been attached to the Surface and Under Ice Trawl (SUIT) net. The SUIT is a trawl developed for sampling meso- and macrofauna in the ice-water interface layer, allowing for greater spatial coverage on the order of a few kilometres \citep{Flores2012, Lange2016, Lange2017}.

In a recent study, \citet{Massicotte2018} showed, that under spatially heterogeneous sea ice and snow surfaces, propagating measured surface downward irradiance just below sea ice \edzerominus{} into the water column using upward attenuation coefficient (\klu{}) calculated from radiance profiles is a better choice compared to the traditional downward vertical attenuation coefficient \ked{} because it is less influenced by surface heterogeneity. However, while the method allows propagation of irradiance to depth from \edzerominus{} more accurately, estimation of representative \edzerominus{} remains difficult. Both ROV and SUIT aim to better describe the horizontal variability of \edzerominus{} under heterogeneous sea ice. Since these technologies are designed to operate at different scales and in different conditions, they are likely to provide complementary information on the light regime experienced by drifting phytoplankton. In this study, we investigated the spatial variability of light transmittance measured from these two devices and combined them with satellite-derived sea ice concentrations. We further used these transmittance data measured at different horizontal spatial scales to quantify how they influence primary production estimates derived from photosynthetic parameters. The results provide new guidance on how to derive more representative primary production estimates under a heterogeneous and changing icescape.