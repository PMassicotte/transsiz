\section{Discussion}

\subsection{Multi-scale spatial variability of light transmittance}

In the context of obtaining meaningful measurements of transmittance to accurately estimate \eparzerosmoins{}, the challenge is to define the spatial extent at which light should be sampled. Based on a spatial autocorrelation analysis conducted in the central Arctic ocean, it was determined that transmittance values were uncorrelated (i.e., randomly spatially distributed) to each other after a horizontal lag distance of 65 m \citep{Lange2017b}. This range is likely to be much smaller than the distance covered by drifting phytoplankton over a 24h period. Indeed, water currents around Svalbard have been found to vary between 0.14 and 0.21 m s\textsuperscript{-1} \citep{Meyer2017} These speeds are on the same order of magnitude as the sea ice drift speeds of 0.10. m s \textsuperscript{-1} observed during the expedition. Assuming passive transport, this corresponds to a displacement varying between 8 and 18 km over a 24h period which is much greater than the 65 m distance at which transmittance was found to be randomly spatially distributed. Under such a large area, drifting phytoplankton is experiencing a wide range of irradiance conditions that can be hardly characterized by single-spot measurements or even with ROV and SUIT devices sampling over larger distances. In such context, measured transmittances should be upscaled at the spatial scale that is meaningful for the studied process. An easily applicable approach to upscale in-situ transmittance measurements consists of using sea ice concentration (SIC) derived from satellite imagery. A simple mixing model (equation 6), combining both in-situ transmittance measurements and SIC, can be used to upscale observations acquired locally to larger scales. Our results showed that using this approach reduced the relative error by approximately a factor of two when spatially integrating devices such as ROVs or SUIT are used to measure transmittance (Figure 5). Furthermore, this error was lower when using in-situ measurements acquired on a larger spatial scale using the SUIT. This strengthens the idea that one needs to characterize the light field over an area as large as reasonably possible so the full irradiance variability is captured.

Our study confirms earlier suggestions that estimating primary production from photosynthetic parameters and transmittance measured at a single location does not provide a representative description of the spatial variability of the primary production occurring under a heterogeneous sea surface (Figure 6, Figure 7). Depending on the scale at which transmittance was measured, it was found that deriving primary production from photosynthetic parameters using under-ice profile  measurements alone would produce on average relative errors varying between 47\% and 88\% (Figure 6). In contrast, much lower errors (25\%) were made when primary production estimates were upscaled using satellite-derived SIC (\ppmixing{}). For stations with lower SIC (stations 19, 27, 31 and 39), primary production estimates were more constrained around the average (Figure 4) because \ppopenwater{} had a greater weight in the calculation of \ppmixing{} (see equation 5). For stations 43, 46 and 47 where SIC was 100\%, the spread around the mean was higher because only \ppunderice{} was contributing to the calculation of \ppmixing{}. These results suggest that using a distribution of measured transmittances allows calculating a more representative transmittance average for a given area, but also provides additional knowledge on its spatial variability.

Although our results indicate that it is necessary to properly characterize the light field under the heterogeneous sea surface, the physiological state of the phytoplankton community also plays a major role on the sensitivity of the estimates to incoming irradiance. An important parameter of the physiological state of the phytoplankton community is the light-saturated photosynthesis regime, $E_k$ an index of photoadaptation. When the phytoplankton community is adapted to low light intensity \citep [e.g.,][]{Lacour2017}, it is likely that variations in the surface light field have reduced impacts on the estimates because phytoplankton primary production is already near saturation. The degree of photoadaptation of the phytoplankton communities and their ability to adjust rapidly to a variable light field still remains to be evaluated.

\subsection{Influence of the number of sampling locations on primary production estimation}

It was pointed out by \citet{Nicolaus2013} that it is difficult to characterize light conditions under sea ice over large areas and to quantify spatial variability on different scales due to important requirements in logistical and instrumental efforts. As with any missions in remote environments such as the Arctic, careful planning is needed to find the right balance between the sampling effort and the right amount of acquired information to study a particular phenomenon. Our results suggested that errors made by estimating primary production using photosynthetic parameters decreased exponentially with increasing number of transmittance measurements (Figure 7). Depending on the extent of the spatial scale at which transmittance is measured (order of meters for the ROV, order of kilometres for the SUIT) and the targeted error thresholds (10\%, 15\%, 20\% or 25\%), a number of measurements varying between four and 359 were sufficient to reasonably capture the spatial variability of sea ice transmittance to derive average primary production estimates over a given area. This shows, that local primary production estimated from just a single or even a handful of light observations has limited value.

\subsection{Implications for Arctic primary production estimates}

It is known that the annual primary production in the ice-covered Arctic is among the lowest of all oceans worldwide because both light and limited nutrient availability are the main limiting factors for phytoplankton growth under the ice. In a changing Arctic icescape, efforts have been devoted to better understand how phytoplankton primary productivity is responding to increasing light availability. Many studies have been conducted in the vicinity of an ice edge to characterize primary production occurring under the ice sheet \citep{Arrigo2012, Arrigo2014, Mundy2009}. However, in such studies, due to logistical constraints, the underwater light field wasoften characterized by a limited number of light measurements. Other approaches, based on 24h ship-board incubations performed under incident light, have provided local estimates that were simply scaled to an assessment of percent ice-cover in the vicinity of the ship \citep{Smith1995, Gosselin1997, Mei2003}. Therefore, depending on whether light is measured under bare ice or in open water, the estimated primary production is either under- or overestimated. Different approaches based on remote sensing techniques and modelling have been used to reduce the high uncertainties associated with estimates derived from local in-situ measurements. However, in an ecosystem model intercomparison study, \citet{Jin2015} showed that under-ice primary production was very sensitive to the light availability computed by atmospheric and sea ice models, reinforcing the need to develop new integrative strategies to adequately characterize the light field at large scale under heterogeneous sea surfaces. Our results showed that upscaling primary production estimates derived from fine-scale local measurements using SIC derived from satellite imagery allowed reducing the error at larger spatial scales. Furthermore, it was found that even when SIC was high (\textgreater 95\%), the use of a mixing-model approach helped to obtain better estimates (Figure 5).