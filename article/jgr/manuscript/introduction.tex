\section{Introduction}
The Arctic sea icescape is characterized by a mosaic composed of sea ice, snow, leads, melt ponds and open water. During the last decades, this arctic icescape has been undergoing major changes, including a reduction of sea ice cover and thickness \citep{Meier2014}, and increased drift speed (Kwok2013). Increasing storm events is also making this icescape more prone to deformation (Itkin2017) and increase the frequency of lead formations. Because of this surface heterogeneity, light transmittance can be highly variable in space, even over short distances (Nicolaus2013b, Katlein2015, Hancke2018). For example, Perovich1998 showed that ice and snow transmittance at 440 nm could vary by a factor of two over horizontal distances of 25 m. The relative contribution of various sea-ice features to under-ice light variability depends on the spatial scale under consideration and has significant implications for their application in physical and ecological studies and also determines the context in which results can be interpreted. For instance, at small scales (< 100 m), local features such as melt ponds and leads have a strong influence on light penetration and fluctuations \citep{Frey2011, Katlein2016, Massicotte2018}. At larger scales (> 100 m), it was argued that the variability of transmittance is mainly controlled by sea ice thickness \citep{Katlein2015}.